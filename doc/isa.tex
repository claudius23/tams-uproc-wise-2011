\section{Instruction set architecture}

\subsection{Data types}

The \cpu uses a word size of 32 bit. All integers are interpreted as
signed, encoded using the two's complement. The word ordering is big
endian.

\subsection{Registers}

There are 32 registers. The first 24 are general purpose registers,
the last eight serve a special role. Each register can be manipulated
using all the instructions that utilize a register.

Each special register has a symbolic name, given in square brackets in
the table below.

\begin{center}
  \begin{tabular}{l|l}
    Register & \\
    \hline
    \%0 - \%23   & general purpose register \\
    \%24 [\%ac]  & accumulator \\
    \%25 [\%pl]  & procedure linkage table \\
    \%26 [\%lr]  & link register \\
    \%27 [\%bp]  & base pointer \\
    \%28 [\%sp]  & stack pointer \\
    \%29 [\%st]  & status register \\
    \%30 [\%pc]  & program counter. The highest bit must be set to 0. \\
    \%31 [\_]    & zero register. Reading from \%31 yields zero, \\
                 & writing to it is ignored. \\
  \end{tabular}
\end{center}

\subsubsection{The accumulator \%ac}
\subsubsection{The procedure linkage table register \%pl}
\subsubsection{The link register \%lr}
\subsubsection{The base pointer register \%bp}
\subsubsection{The stack pointer register \%sp}

\subsubsection{The status register \%st}
\begin{center}
  \begin{tabular}{l|l}
    Bit    & Description \\
    \hline
    31 - 5 & reserved \\
    4      & interupt privilege \\
    3      & secure mode \\
    2      & arithmetic instruction resulted in signed overflow \\
    1      & arithmetic instructions result was negative \\
    0      & arithmetic instructions result was zero \\
  \end{tabular}
\end{center}

\subsubsection{The program counter \%pc}
\subsubsection{The zero register \_}

\subsection{Instruction encoding}

The instruction width is four bytes, the first one (i.e. bits 31-24)
encodes the opcode, the others are used to encode parameters.

\subsubsection{Anatomy of an opcode}

\begin{center}
  \begin{tabular}{c|l}
    Bit & \\
    \hline
    7 - 3  & Instruction selector. \\
    3      & Arithmetic instructions only: Denotes that the status \\
           & register should be updated. \\
    2 - 0  & Conditional selector. \\
  \end{tabular}
\end{center}

\subsubsection{Parameter formats}

\subsection{Memory model}
\subsection{Interupt handling}
\subsection{Security model}
\subsection{Power on state}
