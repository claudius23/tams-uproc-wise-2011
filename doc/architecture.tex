\documentclass{article}
\usepackage[utf8x]{inputenc}
\usepackage{fontenc}
\usepackage{hyperref}
\usepackage{listings}
\usepackage{longtable}
\usepackage[usenames,dvipsnames]{color}

\newcommand{\cpu}{processor}

\author{Claudius, Irina, Justus, Paul}
\title{\cpu\\
tams-uproc-wise-2011}
\date{\today}

\begin{document}

\maketitle
\tableofcontents

\section{ABI}

\subsection{Data types}

The \cpu uses a word size of 32 bit. All integers are interpreted as
signed, encoded using the two's complement. The word ordering is big
endian.

\subsection{Registers}

There are 32 general purpose registers. While some of those serve a
special role, each can be manipulated using all the instructions that
utilize a register.

\begin{center}
  \begin{tabular}{c|l}
    Register & \\
    \hline
    \%0 - \%26 & general purpose register \\
    \%27         & base pointer \\
    \%28         & stack pointer \\
    \%29         & status register \\
    \%30         & program counter. The highest bit must be set to 0. \\
    \%31         & zero register. Reading from zero yields zero, \\
    & writing to it is ignored. '\_' is an alias for \%31. \\
  \end{tabular}
\end{center}

\subsubsection{Control registers}

\subsection{Instruction encoding}

The instruction width is four bytes, the first one (i.e. bits 31-24)
encodes the opcode, the others are used to encode parameters.

\subsection{Opcodes}

\begin{center}
  \begin{tabular}{c|l}
    Bit & \\
    \hline
    7 - 3  & Instruction selector. \\
    3      & Arithmetic instructions only: Denotes that the status \\
           & register should be updated. \\
    2 - 0  & Conditional selector. \\
  \end{tabular}
\end{center}

\subsubsection{List of opcodes}

\include{opcodes}

\end{document}
